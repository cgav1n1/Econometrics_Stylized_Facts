\documentclass{article}
\usepackage{booktabs}
\usepackage{graphicx}
\usepackage{amsmath}
\usepackage[a4paper, total={7in, 10.5in}]{geometry}
\usepackage[toc,page]{appendix}
\usepackage{listings} % Listing package to add appendix
\usepackage{xcolor}   % Custom colors
\definecolor{darkbrown}{rgb}{0.36, 0.2, 0.1}
\usepackage{graphicx}  % Required to include images
\usepackage{hyperref}% to put links
\usepackage{float} %to place correctly pictures
\usepackage{cite} %for better citations

\hypersetup{
    colorlinks=true,
    linkcolor=blue,
    urlcolor=blue,
    citecolor=red,
    linktoc=all,    % Links entire section titles in the ToC
    linkcolor=black % Sets ToC links to black; other links remain as specified
}

% Defining colors for the python section
\definecolor{codegreen}{rgb}{0,0.6,0}
\definecolor{codeblue}{rgb}{0,0,0.8}
\definecolor{codepurple}{rgb}{0.58,0,0.82}
\definecolor{codebackground}{rgb}{0.95,0.95,0.92}

% Suppress page numbering on the first page
\pagenumbering{gobble}

\begin{document}

\title{Mid term Exam for Financial Econometrics with Python}
\author{PRAT Paul; GAVINI Charles; FOURNIER Justin; BLANC Mathieu}
\date{\today}

\maketitle %showcase the title in the top of the page with the informations upper

\tableofcontents %creating a table of contens
\clearpage

% Start counting pages from here
\pagenumbering{Roman} % Preliminary content with roman figures

%first, introduction
\section{Introduction}


This document provides a comprehensive presentation of our results, 
including all relevant tables, figures, and calculations. 
The report is structured into distinct parts, beginning with the importation of essential Python libraries. 
We then initialize variables to organize the data into different categories (e.g., daily, monthly, returns, log returns), 
allowing for clear analysis and comparison across various data types and intervals.


\section{Preliminary}
%Instructions


%At the beginning of your answer to Question 1, indicate clearly:
%- the name and the exact ticker of stock or index you are considering,
%- the length of your sample (number of days, months, years),
%- the initial and final dates of your sample, and
%- the stock market and country in which the stock you selected is traded, or to which the stock index you selected refers to.
%Using the time series you downloaded in PRELIMINARY STEP 1, compute the log-returns at daily, 
%monthly, and annual frequencies and present a table of summary statistics of these three series of returns. 
%The table should be similar to the one appearing in slide n. 91 of the set of slides titled 
%``Lecture 1: Financial Returns: Description and Stylized Facts’’.
\subsection{AMAZON}

The selected stock for this analysis is Amazon due to its significant relevance in current global markets, its impressive growth over time and its position as a major industry leader.
The ticker from yahoo finance is "\textbf{\textit{AMZN}} " on the Nasdaq stock exchange \href{https://finance.yahoo.com/quote/AMZN/.}{AMAZON on Yahoo Finance}
First, importing the Amazon stock with yfinance, then display the pandas table.
We will import 25 years, 8 months and 25 days of data (from 1999-01-21 to 2024-10-16).

\subsection{Data Table}
The data printed here is the preview of the Amazon stock extraction from yahoo finance:
% LaTeX table from 'table.tex'
\begin{table}[h!]
    \centering
    \begin{tabular}{lrrrrrr}
\toprule
{} &      Open &      High &       Low &     Close &  Adj Close &     Volume \\
Date       &           &           &           &           &            &            \\
\midrule
1999-09-16 &  0.051563 &  0.052083 &  0.050260 &  0.050911 &   0.046693 &  158112000 \\
1999-09-17 &  0.050586 &  0.051042 &  0.048958 &  0.050781 &   0.046574 &  171648000 \\
1999-09-20 &  0.049870 &  0.050521 &  0.048958 &  0.048958 &   0.044902 &  229104000 \\
1999-09-21 &  0.047917 &  0.048177 &  0.043620 &  0.044271 &   0.040603 &  737328000 \\
1999-09-22 &  0.044271 &  0.045573 &  0.041667 &  0.045313 &   0.041559 &  375984000 \\
\bottomrule
\end{tabular}
  
    \caption{Preview of Amazon Stock Data (5 first datas) from "AMZN" in Yahoo Finance}
    \label{tab:amazon_stock_preview}
\end{table}



\subsection{Checking the 25 Years range condition}


We need to verify that the data displays accurately over the 25 years range. 
Fortunately, the extracted  Amazon data has been available since January 1999. 
To ensure the data’s continuity and completeness, we will implement a Python script that identifies and counts any gaps within the dataset. 
By visualizing the dates of these gaps, we can easily detect any significant interruptions that could potentially impact our data analysis

\begin{figure}[H]
    \centering
    \includegraphics[width=0.8\textwidth]{Img/MissingDates(2010_to_2011).pdf}
    \caption{Missing Dates in a partial date range (\textbf{01-01-2010} to \textbf{01-01-2011})}
    \label{fig:missing_dates}
\end{figure}

\noindent We identified a total of 238 isolated days of data gaps per year across the 25-years range (6476 values). 
Therefore, the data remains reliable for our stylized facts analysis. 
The missing data points in our dataset are randomly distributed and account for 3.7\% of the total data. 
According to scientific studies on data reliability for volatility testing, a dataset with up to 10% missing data is considered reliable for statistical testing.
\cite{pumi2023longrange} 

\section{First Results}

\subsection{Prices Evolutions}

With the accuracy and the reliability of our dataset confirmed, we begin by plotting the evolution of prices over 4 different periods : Daily, Weekly, Monthly and Yearly prices.

\begin{figure}[H]
    \centering
    \includegraphics[width=\textwidth]{Img/prices_time.pdf}
    \caption{Prices over time $P_t$  by frequency daily, weekly, monthly and annual the AMZN stock.
    Sample: \textbf{01-21-1999} to \textbf{10-16-2024}.}
    \label{fig:prices_time}
\end{figure}

\subsection{Calculating Returns}
Using the processed data, we can now output graphs for several key metrics: 
daily prices, daily log prices, daily simple returns, and daily log returns. 
Plotting these metrics will allow us to observe daily price movements, 
the transformation of prices into $log$ form for trend analysis, 
as well as daily returns and their logarithmic equivalents.
\begin{figure}[H]
    \centering
    \includegraphics[width=\textwidth]{Img/log_returns.pdf}
    \caption{Prices $P_t$, returns $R_t$ and log returns $r_t$ of the AMZN stock.
    Sample: \textbf{01-21-1999} to \textbf{10-16-2024}.}
    \label{fig:log_returns}
\end{figure}

\subsection{Squared Returns}

To complete the analysis of daily price data, we also plot the daily squared returns and daily squared log returns, 
(providing us a key insight on the volatile behavior on the potential stock risk of our data set.)
\begin{figure}[H]
    \centering
    \includegraphics[width=0.8\textwidth]{Img/squared_log_returns.pdf}
    \caption{Squared daily returns $R_t^2$ and daily squared log returns $r_t^2$ of the AMZN stock.
    Sample: \textbf{01-21-1999} to \textbf{10-16-2024}.}
    \label{fig:squared_logreturns}
\end{figure}

\clearpage
% Start main content with Arabic numbering from 1
\pagenumbering{arabic}
\setcounter{page}{1}

\section{Amazon and the 8 Stylized Facts}

\subsubsection{Summary statistics}


\begin{table}[H]
    \centering
    \begin{tabular}{lrrrr}
\toprule
{} &        daily &      weekly &    monthly &     annual \\
\midrule
Mean                    &      0.06551 &     0.28602 &    1.34694 &   15.59259 \\
St.Deviation            &      3.27670 &     6.78240 &   13.04915 &   56.87533 \\
Diameter.C.I.Mean       &      0.07973 &     0.36248 &    1.45499 &   22.29513 \\
Skewness                &      0.39404 &     0.04813 &   -0.46401 &   -0.99033 \\
Kurtosis                &     11.04424 &     7.51963 &    2.60379 &    1.46124 \\
Excess.Kurtosis         &      8.04424 &     4.51963 &   -0.39621 &   -1.53876 \\
Min                     &    -28.45678 &   -38.51804 &  -53.02674 & -158.75126 \\
Quant5                  &     -4.64852 &    -9.91694 &  -20.10778 &  -66.66688 \\
Quant25                 &     -1.26082 &    -2.66403 &   -4.96881 &  -17.23374 \\
Median                  &      0.04153 &     0.30015 &    2.12289 &   21.13060 \\
Quant75                 &      1.39904 &     3.40521 &    8.48954 &   55.72364 \\
Quant95                 &      4.50385 &    10.67205 &   20.87667 &   94.22164 \\
Max                     &     29.61811 &    56.11507 &   48.35221 &  102.44636 \\
Jarque.Bera.stat        &  33141.87111 &  3169.38203 &   98.37746 &    6.31063 \\
Jarque.Bera.pvalue.X100 &      0.00000 &     0.00000 &    0.00000 &    4.26249 \\
Lillie.test.stat        &      0.10370 &     0.09576 &    0.08174 &    0.09522 \\
Lillie.test.pvalue.X100 &      0.10000 &     0.10000 &    0.10000 &   79.74330 \\
N.obs                   &   6488.00000 &  1345.00000 &  309.00000 &   25.00000 \\
\bottomrule
\end{tabular}
  
    \caption{Summary statistics for the AMZN stock.
    Sample: \textbf{01-21-1999} to \textbf{10-16-2024}.}
    \label{tab:Stylized_facts_preview}
\end{table}


% refering to the table with : Table~\ref{tab:Stylized_facts_preview}

\subsection{Prices are non-stationary}

The first feature that will highlight non-stationarity of
the prices is the comparison of \( p_t \) vs \( p_{t-1} \).

\begin{figure}[H]
    \centering
    \includegraphics[width=0.5\textwidth]{Img/Laggedlog(p_t-1).pdf}
    \caption{Comparison of \( \log(p_t) \) vs \( \log(p_{t-1}) \) of the AMZN stock.
    Sample: \textbf{01-21-1999} to \textbf{10-16-2024}.}
    \label{fig:LogptVSLogpt-1}
\end{figure}

\noindent The graph in Figure \ref{fig:LogptVSLogpt-1} demonstrates this strong linear 
relationship, indicating that Amazon's prices at time \( t \) are highly dependent on those at \( t-1 \) and lack mean reversion,
 supporting the idea of non-stationarity.

\noindent Additionally, the empirical autocorrelation function (ACF) of Amazon's daily prices shows a slow decay, further suggesting non-stationarity, as shown in the next figure.

\begin{figure}[H]
    \centering
    \includegraphics[width=0.8\textwidth]{Img/Autocorrel_daily_monthly.pdf}
    \caption{Autocorrelations of daily and monthly prices of the AMZN stock.
    Sample: \textbf{01-21-1999} to \textbf{10-16-2024}.}

    \label{fig:Autocorrelations_daily_monthly}
\end{figure}

\noindent For the amzon daily and monthly prices time series, we expect to see large values of $\hat{\rho}_k$, i.e., near to 1, slowly decaying as $k$ increases this is the \textbf{long memory property}.
Also, the \textbf{slowly decaying ACF} is often a symptom of non stationarity.
\subsection{Returns are stationary}

\begin{figure}[H]
    \centering
    \includegraphics[width=0.5\textwidth]{Img/Daily_Log_Returns.pdf}
    \caption{Daily Log-returns $r_t := p_t - p_{t-1}$ of the AMZN stock.
    Sample: \textbf{01-21-1999} to \textbf{10-16-2024}.}
    \label{fig:Daily_log_returns}
\end{figure}

\noindent $Log$-returns are a common way to measure the percentage change in stock prices, 
and they help assess the stability or stationarity of the returns over time. In a stationary series, we would expect the properties, 
such as mean and variance, to remain constant over time. 
However, here we observe significant differences in volatility across the timeline. \\

\noindent In the early years (around 1999-2005), there is noticeably higher volatility in Amazon's log-returns, 
with frequent large spikes both upwards and downwards. This period corresponds to the tech boom and subsequent dot-com bubble burst, 
during which many tech stocks, including Amazon, experienced extreme price fluctuations. 
Additionally, as a relatively new and fast-growing company, Amazon’s stock likely faced higher uncertainty and speculative trading, 
contributing to greater volatility.

\subsection{Asymmetry}

\begin{figure}[H]
    \centering
    \begin{minipage}{0.45\textwidth}
        \begin{table}[H]
            \centering
            \begin{tabular}{lrrrr}
\toprule
{} &     daily &   weekly &  monthly &   annual \\
\midrule
Skewness &   0.39404 &  0.04813 & -0.46401 & -0.99033 \\
Kurtosis &  11.04424 &  7.51963 &  2.60379 &  1.46124 \\
\bottomrule
\end{tabular}
  
            \caption{\textbf{Skewness and kurtosis of daily, weekly, monthly and annual} $log$ returns of the AMZN stock. 
            Sample: \textbf{01-21-1999} to \textbf{10-16-2024}.}
            \label{tab:skewness_kurtosis}
        \end{table}
    \end{minipage}
    \hspace{0.05\textwidth}
    \begin{minipage}{0.45\textwidth}
        \centering
        \includegraphics[width=0.8\textwidth]{Img/Fact3_2_rollskew.pdf}
        \caption{\textbf{Rolling skewness} of AMZN stock. 
        Sample: \textbf{01-21-1999} to \textbf{10-16-2024}. The \textcolor{red}{red bands} corresponds to the limit of acceptance, the blue line correspond to the rolling skewness with T=252}
        \label{fig:Rolling_skewness}
    \end{minipage}
\end{figure}

\noindent For this case, Table~\ref{tab:skewness_kurtosis} hilights that for daily returns the AMZN stock skewness is positive.
This does not confirms stylized fact 3, this case is not really common but it means that the mean return is higher than the median of the sample \cite{albuquerque2012skewness}. 
Then, Amazon investors tend to have steeper high turns than downturns and that investors of the AMZN stock react more positively to good news than they can react badly for negative news.
If we take a look at the Rolling Skewness on simple returns Figure~\ref{fig:Rolling_skewness}, we clearly see that the skewness (for a 252 days interval) varies a lot depending the position of intervals and has already been very negative (Dec 2004) but is generally positive.
\subsection{Heavy tails}
\noindent As showcased in the Table~\ref{tab:skewness_kurtosis}, there is a large excess kurtosis

\begin{figure}[H]
    \centering
    \includegraphics[width=0.6\textwidth]{Img/QQplot_daily_weekly_AMZN.pdf}
    \caption{Log returns $r_t := p_t - p_{t-1}$: \textbf{histograms of daily, monthly} “adjusted closing” of AMAZON. 
    Sample: \textbf{01-21-1999} to \textbf{10-16-2024}. QQ plot against quantiles of normal distribution with same mean and variance as the empirical distribution of returns.}
    \label{fig:Hstogram_QQ_plot}
\end{figure}

\noindent Here, the QQ-Plots explicit clearly how our sample distinguishes from the normal distribution. 
The QQ-plot provide graphical evidence that the tails of the daily returns distributions are heavier than the tails of the normal distribution as: The points on the left of the graph which represent the lower quantiles (i.e. the points in the left tail of the empirical distribution) are below the blue line. The lower quantiles of the empirical distribution are much smaller than what you should expect from a Normal random variable with the same empirical mean and standard deviation of the sample the left tail of the empirical distribution is heavier than one of a Normal Distribution. similar conclusions for the right tail.
\begin{figure}[H]
    \centering
    \includegraphics[width=1\textwidth]{Img/qqplt_tstudents_AMZNdaily.pdf}
    \caption{Log returns $r_t := p_t - p_{t-1}$: \textbf{daily} “adjusted closing” of AMZN stock. 
    Sample: \textbf{01-21-1999} to \textbf{10-16-2024}. \textbf{QQ plot of Sample standardized quantiles} (0 mean and unit variance) \textbf{of daily log-returns against quantiles of standardized} (0 mean and unit variance) \textbf{Student-t distributions} with $\nu = 10$, 5, and 3 degrees of freedom.}
    \label{fig:Hstogram_QQ_plot_T_student}
\end{figure}
\subsection{Gaussianity}
\subsubsection{High frequency non-Gaussianity}

The aggregate gaussianity, states that lower frequency returns (monthly) tend to be Gaussian (symmetric about the mean) even if higher frequency returns (daily) are not. 
To test this stylized fact we perform a Jarge-Bera test. The result is in Table~\ref{tab:Stylized_facts_preview}. \\

\noindent The 3\textsuperscript{rd} central moment is defined as
$\mu_3 := E((X - m_1)^3).$ The skewness of \( r_t \) is defined as:
\[
\text{Skew}(r_t) := E \left[ \left( \frac{X - m_1}{\sigma} \right)^3 \right] = \frac{\mu_3}{\sigma^3} = \frac{\mu_3}{\mu_2^{3/2}}.
\]

\noindent As the result in Table~\ref{tab:Stylized_facts_preview} 
the skewness is positive for daily and monthly data, \( \text{Skew}(r_t) >0 \), large realizations of \( X \) are more often larger
than the mean \( \mu \). 
Skewness is thus used as a measure of asymmetry of the distribution \( f_X(x) \). Therefore:
 
\( \text{Skew}(r_t) > 0 \), so the distribution is said to be \textbf{right skewed}. 
%For symmetric distributions (e.g., Gaussian, t-Student, uniform), 
%we have \( \text{Skew} = 0 \). In these cases, all odd moments are zero:
%\( \mu_r = E[(X - \mu)^3] = 0 \) for \( r = 3, 5, \dots \)

\( \text{Skew}(r_t) > 0 \), then \( \mu > \text{median} \).

\subsubsection{Aggregational Gaussianity}
\begin{figure}[H]
    \centering
    \includegraphics[width=0.8\textwidth]{Img/lillie_test_AMZNannualy.pdf}
    \caption{Log returns $r_t := p_t - p_{t-1}$: \textbf{annual} “adjusted closing” of AMZN. 
    Sample: \textbf{01-21-1999} to \textbf{10-16-2024}. \textbf{Left panel}: empirical and Normal cdf's for the standardized annual returns of AMZN. \textbf{Right panel}: values of $\left| G_T(\tilde{r}_t) - \Phi(\tilde{r}_t, \hat{\mu}, \hat{\sigma}^2) \right|$ (\textcolor{blue}{blue line}) and critical values for the Lilliefors test for the three significance levels \textcolor{orange}{10\%}, \textcolor{red}{5\%} and \textcolor{darkbrown}{1\%}.}
    \label{fig:Lillie_test_weekly}
\end{figure}

\noindent The blue line is under the critical values lines, 
So the test is respected and so for the Gaussianity.
\subsection{Returns are not autocorrelated}

Stylised fact 6 posits that returns are not autocorrelated. 
Autocorrelation in a weakly stationary process measures the correlation between values of the process at different time points. 
To assess the significance of autocorrelations, we apply the \textbf{Box-Pierce (BP)} test or the \textbf{Ljung-Box (LB)} test, 
where the null hypothesis indicates that all autocorrelations are equal to 0, compared with the alternative analysis where it differs from 0.

\begin{table}[H]

    \centering
    \begin{tabular}{lrrrrrrrrr}
\toprule
{} &  lag &    acf &  acf diam. &  acf test &  B-P stat &  B-P pval &  L-B stat &  L-B pval &    crit \\
\midrule
0  &    1 &  0.054 &      0.111 &     0.950 &     0.903 &     0.342 &     0.911 &     0.340 &   3.841 \\
4  &    5 &  0.022 &      0.111 &     0.386 &     8.178 &     0.147 &     8.329 &     0.139 &  11.070 \\
14 &   15 & -0.052 &      0.111 &    -0.917 &    23.761 &     0.069 &    24.481 &     0.057 &  24.996 \\
24 &   25 & -0.056 &      0.111 &    -0.983 &    38.351 &     0.043 &    40.250 &     0.027 &  37.652 \\
\bottomrule
\end{tabular}
  
    \caption{Ljung-Box and Box-Pierce daily}
    \label{tab:LB_BP}
\end{table}
In Table~\ref{tab:LB_BP}, we observe that for each lag (1, 5, 15 and 25), the p-values for both the \textbf{Ljung-Box (L-B pval)} and \textbf{Box-Pierce} ($B-P pval$) tests exceed the \textbf{$0.048$ threshold}. This indicates that there is insufficient statistical evidence to reject the null hypothesis at each lag. Consequently, this suggests that the returns are not significantly autocorrelated across these time lags.
\subsection{Volatility clustering and long range dependence of squared returns}

Volatility clustering is a phenomenon where periods of high market volatility are often followed by high volatility, and vice versa. To capture and analyze this phenomenon, financial models such as ARCH (Autoregressive Conditional Heteroskedasticity) and GARCH are commonly used. We can easily perceive it on the graph below, (from december 2001 to december 2004) phase of low volatility.

\begin{figure}[H]
    \centering
    \includegraphics[width=0.5\textwidth]{Img/Fact7_AMZN_rolling_stdev.pdf}
    \caption{Rolling standard deviation from the “adjusted closing” of AMZN. Sample: \textbf{01-21-1999} to \textbf{10-16-2024}.}
    \label{fig:Rolling_std_dev_1}
\end{figure}

\noindent This persistence in the autocorrelation of squared returns reflects volatility clustering. 
High volatility often persists over time before settling into a lower volatility regime; 
this is how time dependance is reflected.



\begin{figure}[H]
    \centering
    \includegraphics[width=1\textwidth]{Img/Fact7_AbsoluteLogReturns.pdf}
    \caption{Autocorrelations of the daily, weekly and monthly absolute log-returns $r_t$ from the “adjusted closing” of AMZN. Sample: \textbf{01-21-1999} to \textbf{10-16-2024}.
    The \textcolor{blue}{blue dotted} bands represents the confidence intervals (\textbf{Barlett intervals}), $\frac{1}{\sqrt{T}}$ where T is the number of samples.}
    \label{fig:atocorrelation_abs_logreturns}
\end{figure}

\noindent We observe that the autocorrelation is continuous, as indicated by the trendline, 
which aligns with the previous graph. 
Additionally, it becomes apparent that as the time interval changes (from daily to weekly, monthly, and annually) 
the autocorrelation becomes more pronounced between intervals. 
This aligns with the volatility clustering phenomenon discussed earlier. 
This effect occurs because ARCH and GARCH models are sensitive to sampling frequency, 
with their impact being more noticeable at shorter frequencies (daily, weekly) than at longer ones (monthly, annually).

\subsection{Leverage effect}

\begin{figure}[H]
    \centering
    \includegraphics[width=0.5\textwidth]{Img/Fact8_CrossCorr_r_r2.pdf}
    \caption{Rolling standard deviation from the “adjusted closing” of AMZN. Sample: \textbf{01-21-1999} to \textbf{10-16-2024}.}
    \label{fig:Rolling_std_dev_2}
\end{figure}

\begin{figure}[H]
    \centering
    \includegraphics[width=0.5\textwidth]{Img/Fact8.pdf}
    \caption{Rolling standard deviation from the “adjusted closing” of AMZN. Sample: \textbf{01-21-1999} to \textbf{10-16-2024}.}
    \label{fig:Rolling_std_dev_3}
\end{figure}

\begin{figure}[H]
    \centering
    \includegraphics[width=0.5\textwidth]{Img/Fact_8_3AMZN_.pdf}
    \caption{Rolling standard deviation from the “adjusted closing” of AMZN. Sample: \textbf{01-21-1999} to \textbf{10-16-2024}.}
    \label{fig:Rolling_std_dev}
\end{figure}

\appendix

\lstset{
    language=Python,
    backgroundcolor=\color{codebackground},   % Background color
    basicstyle=\ttfamily\footnotesize,        % Font and size
    breaklines=true,                          % Automatic line breaking
    frame=single,                             % Frame around code
    numbers=left,                             % Line numbers on the left
    numberstyle=\tiny\color{gray},            % Line number style
    keywordstyle=\color{codeblue}\bfseries,   % Keyword color
    commentstyle=\color{codegreen}\itshape,   % Comment color
    stringstyle=\color{codepurple},           % String color
    showstringspaces=false,                   % Hide string spaces
    captionpos=b,                             % Caption position: bottom
    tabsize=4                                 % Tab width
}

\section{Appendix: Python Code}
Below is the Python code used in the analysis.

\begin{lstlisting}[caption=Python Code for Analysis]
# Python code example
import numpy as np
import pandas as pd

def analyze_data(data):
    mean = np.mean(data)
    std_dev = np.std(data)
    return mean, std_dev

data = [1, 2, 3, 4, 5]
mean, std_dev = analyze_data(data)
print(f"Mean: {mean}, Standard Deviation: {std_dev}")
\end{lstlisting}
  % Include the code appendix

\section{To go further, CAPM pricing model}

\bibliographystyle{plain}
\bibliography{bibliography} % Do not include the .bib extension


\end{document}